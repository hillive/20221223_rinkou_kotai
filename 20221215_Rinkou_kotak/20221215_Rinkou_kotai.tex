% twocolumn を使うと2段組になる

%\documentclass[a4j,twocolumn]{jsarticle}        % -> platex
%\documentclass[a4j,twocolumn]{ujarticle}       % -> uplatex
\documentclass[uplatex]{jsarticle}   % -> uplatex + jsarticle

\usepackage{resume} % 他パッケージ,専用コマンド,余白の設定が書かれている

\kcatcode"3000=15

\usepackage{newunicodechar}

\newunicodechar{。}{\char"FF0E}
\newunicodechar{、}{\char"FF0C}
%%%%%%%%%%%%%%%%%%%%%%%%%%%%%%%%%%%%%%%%%%%%%%%%%%%%%%%%%%%%%%%%%%%%%%%%
% ヘッダ: イベント名,日付,所属,タイトル,氏名
%%%%%%%%%%%%%%%%%%%%%%%%%%%%%%%%%%%%%%%%%%%%%%%%%%%%%%%%%%%%%%%%%%%%%%%%

\pagestyle{plain}
\newcommand{\comment}[1]{}
\begin{document}
\twocolumn[
\beginheader{令和4年度 コンピュータサイエンス学部 輪講発表}{2022}{12}{15}{井上 研究室}
\title{HMD提示によるベクション刺激と嗅覚刺激の知覚的相互作用に関する検討}
\author{C0B20020 市川 晃大 (Kota Ichikawa) }
\endheader
]

\vspace{3mm}

 % 本番用ページ番号オフセット
\setcounter{page}{5}

%---------------------------------------------------------------------------
% 本文
%---------------------------------------------------------------------------


\section{はじめに}
ベクションとは狭義には視覚刺激によって
駆動される自己移動感覚のことであり、ウェアラブル機器であるヘッドマウントディスプレイ(HMD)を用いたVRシステムなどでも頻繁に誘発される\cite{翔悟17}。
またベクションには人間理解のための基礎研究として知覚心理学的な価値がある。ベクションを研究することは多感覚研究の一翼を担うことになる。

嗅覚刺激については、VRシステムを使用した映像やゲームなどに同期させて提示する試みが多くなされており、映像に適した香りを付加することは今後のVRシステムにおける錯覚的な自己移動感覚(すなわちベクション)の促進法として有効である。

そこで本研究では香り刺激強度とベクション刺激強度の相互作用を再検討するために、提示する刺激の強度と種類を変化させ、ベクションと嗅覚の相互作用を調査した。

\section{関連研究}
\subsection{ベクションと多感覚相互作用}
聴覚刺激や触覚刺激は、それ自体でベクションを引き起こす以外にも、視覚刺激と共に提示することで、視覚性ベクションの知覚を促進することが分かっている。\cite{翔悟20}は視覚的ランドマークと一致する音が視覚刺激に合わせて移動することで、仮想環境での視覚性回転ベクションと自己プレゼンスの両方を促進することを報告している。におい源を空間的に局在化するのは困難だが、可能である\cite{翔悟21}。そのため、嗅覚刺激が視覚性ベクション知覚を促進する可能性もあるが、それらは全く調査されていない。

\section{実験概要}
そこで、ベクションと嗅覚の相互作用について調査する。そのために、嗅覚ディスプレイによる香り提示下で、HMDを装着してベクション生起する動画を見てもらう。そして、ベクションと香りの主観強度を評価させた。本実験には、一般的な嗅力を有している学部生、大学院生の男性、全10名が参加した。平均年齢が21.9歳であった。

\section{システム概要}
本実験では、香りの提示に\figref{fig:inkjet}インクジェット方式嗅覚ディスプレイFragrance Jet 2を用いた。これはインクタンクから香料を微小な液滴として空気中に射出する。提示された香料は極めて微細であるため、その大部分が気化する。そして気化した香料は嗅覚ディスプレイ後部にファンからの風により、被験者の鼻元へ運ばれる。

\begin{figure}[htbp]
 \centering
 \includegraphics[clip,width=7cm]{./inkjet.png}
 \caption{実験の様子}\label{fig:inkjet}
\end{figure}

\subsection{実験手順}
ベクション刺激の提示時間は40秒とした。動画再生中は
画面中央辺りをぼんやりと見るように被験者に教示し、
実験中は鼻で自然な呼吸を行うように指示した。被験者には、
ベクション刺激を観察中に、持続時間、マグニチュードを
報告させた。さらに、表1の6段階臭気強度表示からの香りの
主観的な強度を選択させた。

\begin{table}[htbp]
 \centering
 \caption{6段階臭気強度表示法}
% \label{tab:fuga}
 \begin{tabular}{|c|c|}\hline
  臭気強度 & 内容 \\ \hline
  0 & 無臭 \\ \hline
  1 & やっと感知できるにおい\\
  & (検知閾値に相当)\\ \hline
  2 & 何のにおいかがわかる弱いにおい \\
  & (認知閾値に相当)\\ \hline
  3 & 楽に感知できるにおい \\ \hline
  4 & 強いにおい \\ \hline
  5 & 強烈なにおい \\ \hline
 \end{tabular}
\end{table}

被験者をドットの速度・密度が大きいベクション強グループと速度・密度が小さいベクション弱グループに分け、一人当たり8条件(ベクション2種類×香り4種類)の実験を行った。

被験者を嗅覚ディスプレイが置かれている机の前に座らせ、
顎を顎乗せ台に乗せた状態で行った。被験者には常に
HMDとヘッドフォンを装着させたが、音刺激は提示しなかった。嗅覚ディスプレイ内部射出口から鼻までの距離は225mmに固定した。ディスプレイの提示口を斜め上に向けて設置し、被験者ごとに顎の高さの調整も行った。実験中は、香り提示時以外も常にファンを起動させ、風によるベクション知覚への影響の差をなくした。風速は約1.8m/秒であった。実験の様子を\figref{fig:jikken}に示す。

\begin{figure}[htbp]
 \centering
 \includegraphics[clip,width=7cm]{./jikken.png}
 \caption{実験の様子}\label{fig:jikken}
\end{figure}

\section{結果}
\subsection{ベクションが香り知覚へ与える影響}
\label{ベクションが香り知覚へ与える影響}
被験者全体で、2種類の香料それぞれについて濃度の強弱条件、2種類のベクション刺激それぞれについて速度・密度の大小による強度条件、計16条件での実験を行い、全ての組み合わせについてデータを得た。ベクション方向要因、ベクション強度要因、香り種類要因、香り強度要因それぞれについて、Welch検定を実施したところ、ベクション強度要因$(t(309.41)=2.19, P < 0.05)$と香り強度要因$(t(318)=2.65, P < 0.01)$で有意性が見られた。また、香りの知覚強度を従属変数に、速度・密度大小によるベクションの強弱を説明変数にとった。そして香り刺激強の場合と香り刺激弱の場合についてその他の条件のデータをプールして平均値を算出した。香り強条件の方が香り弱条件より香りを強く感じていること、ベクションの速度と密度が大きくなると香りの知覚強度も上がることが確認できた。

\subsection{香りがベクション知覚へ与える影響}
続いて、香りからベクション知覚への影響を調べた。\ref{ベクションが香り知覚へ与える影響}節と同様に、ベクション方向要因、ベクション強度要因、香り種類要因、香り強度要因それぞれにおけるベクション3指標に対してWelch検定を実施したところ、ベクション方向要因の潜時$(t(282.12)=2.01, P < 0.05)$とマグニチュード$(t(318)=2.35, P < 0.05)$で有意義が見られた。

ベクション方向別にベクション強度要因と香り有無要因で、Welch検定を実施した。拡散刺激、収束刺激どちらにおいても、マグニチュードにおけるベクション強度要因で有意差が見られた(拡散:$t(254)=2.10, P<0.05$, 収束:$t(254)=2.79, P<0.01$)。

\section{おわりに}
本研究では、提示するベクション刺激の速度・密度と香り強度を変化させ、ベクションと嗅覚の相互作用に与える影響を調査した。その結果、2つの感覚間に相互作用があることが明らかになった。特にベクション刺激が嗅覚へ与える影響は明確に見られ、強く示唆される結果となったことは重要と考える。

今後は、より多くの種類の刺激を用いた場合のベクション刺激と嗅覚の相互作用についても調査していく必要がある。

今回得られた知見は、ベクション感覚と香り知覚との関係を明らかにするという知覚心理学での基礎研究的価値のみならず、VRでの応用価値も有することが期待できる。



%---------------------------------------------------------------------------
% 本文終わり
%---------------------------------------------------------------------------

 % 参考文献
\begin{thebibliography}{99}
  \bibitem{翔悟17}Palmisano,S.,Allison,S,R,Schira,M.M.,and Barry,J.R.:Future challenges for vection research:definitions,functional significance,measures,and neural bases;Front Psychol,6(PMC4342884),(2015.2)
%  \bibitem{翔悟18}Palmisano, S., Mursic, R., Kim, Juno.:Vection and cyversickness generated by head-and-display motion int the Oculus Rift; Display, 46, 1-8(2017.1)
%  \bibitem{翔悟19}Kim,J.,Chung,YL,C.,Nakamura,Shinji.,Palmisano,\\S.,Khuu,K,S.:The Oculus Rift:a cost-effective tool for studying visual-vestibular interactions in self-motion perception;Frontiers in psychology,6,248(2015.3)
  \bibitem{翔悟20}Seno,T.,Ogawa,M.,Ito,H.,Sunaga, S.:Consistent air flow to the face facilitates vection;Perception,40,1237-1240(2011.1)
  \bibitem{翔悟21}Welge-Lussen,A.,Looser,GL.,Westermann,B.,Hummel,T.:\\Olfactory source localization in the open field using one or both nostrils;Rhinology,52(1),41-47(2014.3)
\end{thebibliography}
 
%\bibliographystyle{junsrt}
%\bibliography{ref}


\end{document}


%-----------------------------------------------------
% テンプレート
%------------------------------------------------------------------------------

%-----------
%% 箇条書き
%-----------
%\begin{itemize}
% \item
%\end{itemize}

%-------------------
%% 番号付き箇条書き
%-------------------
%\begin{enumerate}
% \item
%\end{enumerate}

%-----------
%% 図の表示
%-----------
%\begin{figure}[H]
% \centering
% \includegraphics[clip,width=7cm]{hoge.eps}
% \caption{図タイトル}\label{fig:hoge}
%\end{figure}

%-----------
%% 図の参照
%-----------
%\figref{fig:hoge}

%-----------
%% 表の作成
%-----------
%\begin{table}[H]
% \centering
% \caption{表タイトル}\label{tab:fuga}
% \begin{tabular}{|c|c|c|}\hline
%  hemo & piyo & fuga \\ \hline
%  hemo & piyo & fuga \\ \hline
% \end{tabular}
%\end{table}

%-----------
%% 表の参照
%-----------
%\tabref{tab:fuga}

%-----------
%% 参考文献
%-----------
%\begin{thebibliography}{9}
% \bibitem{piyo} 参考文献
%\end{thebibliography}

%-----------------
%% 参考文献の参照
%-----------------
%\cite{piyo}